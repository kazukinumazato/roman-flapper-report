\section{MODELING}
\label{sec:modeling}

\begin{figure}[!b]
    \centering
    \includegraphics[width=\columnwidth]{modeling.png}
    \caption{The mechanical structure model of a tailless aerial robotic flapper.
    The upper part shows the drone in the from above, and the lower part shows the drone in the front view.
    It has motors for thrust (m1, m2), a motor for pitch control (m3), and a motor for yaw control (m4).
    The wings make flapping motion while m1 and m2 is rotating.
    }
    \label{figure:modeling}
  \end{figure}

In this section, we describe the dynamic model and control of a basic flapping-wing drone.
There are various types of flapping-wing drones, and we focus on a tailless aerial robotic flapper model~\cite{karasek2018tailless}, as shown in Fig.~\ref{figure:modeling}.
The drone is mounted with four motors: two motors for thrust (m1, m2), one motor for pitch control (m3), and one motor for yaw control (m4).
We can treat the dynamics as a single rigid body as follows:
\begin{equation}
    \begin{aligned}
      &m\bm{a} = m\bm{g} + \bm{\mathit{R}}\bm{f} \\
      &\bm{\mathit{I}}\dot{\bm{\omega}} + \bm{\omega} \times \bm{\mathit{I}}\bm{\omega} = \bm{\tau}
    \end{aligned}
\end{equation}
where $m$ is the mass of the drone, 
$\bm{a}$ is the acceleration of the drone,
$\bm{g}$ is the gravity vector,
$\bm{\mathit{R}}$ is the orientation matrix of the drone,
$\bm{f}$ is the thrust force of the drone,
$\bm{\mathit{I}}$ is the inertia matrix of the drone, 
\bm{$\omega$} is the angular velocity of the drone, 
and \bm{$\tau$} is the torque applied to the drone.
The thrust force $\bm{f}$ and the torque $\bm{\tau}$ can be represented as follows:
\begin{equation}
  \label{eq:control}
  \begin{aligned}
    \begin{bmatrix}
      \bm{f}\\
      \bm{\tau}
    \end{bmatrix}
    &=
    h(\omega_1, \omega_2, \theta_3, \theta_4)\\
  \end{aligned}
\end{equation}
where $\omega_i$ is the angular velocity of the motor $i$, 
$\theta_i$ is the angle of the motor $i$, 
and $h$ is the mapping function from $\mathbb{R}^4$ to $\mathbb{R}^6$.
Based on these equations, we can derive the following PID control law:
\begin{equation}
  \begin{aligned}
    &\bm{f}_d = {PID}_r(\bm{e}_r)
    &\bm{\tau}_d &= {PID}_R(\bm{e_\theta})\\
    &\bm{e_r} = \bm{r} - \bm{r}_d
    &\bm{e_\theta} &= \bm{\theta} - \bm{\theta}_d\\
  \end{aligned}
\end{equation}
where $\bm{f}_d$ is the desired thrust force, 
$\bm{\tau}_d$ is the desired torque, 
$PID_r$ is the PID controller for the position, 
$PID_R$ is the PID controller for the orientation, 
$\bm{e_r}$ is the error of the position, 
$\bm{e_\theta}$ is the error of the orientation, 
$\bm{\theta}$ is the orientation of the drone, 
and $\bm{\theta}_d$ is the desired orientation of the drone.
Then, we can calculate the desired motor inputs from the desired thrust force and torque using the mapping function $h$ in (\ref{eq:control}).

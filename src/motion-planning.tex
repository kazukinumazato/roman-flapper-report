\section{MOTION PLANNING}

To design a motion planning method that minimizes psychological discomfort when a flapping drone approaches a human, we must consider the several factors:

\subsection{Distance}

Given the expected size and shape of a flapping drone, we estimate that the minimum acceptable distance falls within a specific range. By maintaining this distance while gradually invading the landing zone, the psychological burden can be minimized.
Several studies have investigated the psychological burden imposed by drones depending on their distance from humans useing different drone sizes\cite{Yeh2017Proxemics, lieser2021evaluating-distances,Duncan2013comfortable-approach, Acharya2017robot-vs-drone-comfort}.
\cite{lieser2021evaluating-distances} is a paper focusing on tactile drone interaction using a drone with a wheelbase of 0.92m, which is suitable for palm landing.
This paper conducted an experiment in which participants were asked to stop the drone when they felt uncomfortable by using foot (non-contact) and hand (contact) methods.
The results show that the overall stop distance was $\bar{x}$ = 0.63m ($\sigma$ = 0.33m),
which means that, if the drone needs to approach closer than this distance, it should be done in psychologically safe manners such as gradually decreasing the speed or using a trajectory that does not directly approach the user so that the user does not feel threatened.

Moreover, the study \cite{lieser2021evaluating-distances} also shows that the average distance at which participants stopped the drone using their hands was $\bar{x}$ = 0.47m ($\sigma$ = 0.09m).
This indicates that physical contact should occur outside this range to guarantee physical and psychological safety.

\subsection{Altitude}

According to \cite{mdpi_2023}, an appropriate altitude for approach is around X meters, as excessively high approaches can lead to discomfort.

\subsection{Approach Direction}

As indicated in \cite{lieser2021evaluating-distances}, drones positioned behind a person induce greater psychological burden. Therefore, the drone should approach either from the user's front or from the left side to ensure comfort.

\subsection{Velocity}

The perception of speed follows Weber's Law: at closer distances, small changes in distance appear faster, whereas at greater distances, motion perception is less sensitive \cite{weber_law}.

Thus, slower speeds are preferable at closer distances. Studies on driver perception of approaching vehicles \cite{jsaeronbun_42_2_619} suggest that gradual deceleration is beneficial, as it reduces overshooting when the drone reaches the hand position.

To further decrease the sense of intrusion, it is preferable for the drone to descend onto the palm from a direction perpendicular to the user's line of sight. This necessitates dividing the trajectory into distinct approach phases.

\subsection{Trajectory Design}

Based on the above findings, we propose the following trajectory, as illustrated in Figure \ref{fig:trajectory}.

\begin{figure}[t]
    \centering
    \includegraphics[width=\columnwidth]{dj.jpg}
    \caption{Proposed trajectory for flapping drone approach.}
    \label{fig:trajectory}
\end{figure}



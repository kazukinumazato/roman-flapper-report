\section{Introduction}
In recent years, extensive research has been conducted on Human-Drone Interaction (HDI), exploring various applications. 
In particular, physical contact-based interaction between drones and humans has gained increasing attention, as it has the potential to expand the scope of drone applications such as \cite{Knierim2017virtual-reality-tactile-drones, Nitta2014hoverball}. 
However, ensuring physical and psychological safety during physical contact has been a significant challenge, especially when using conventional propeller-driven drones.  
The rapid rotation of propellers poses a potential risk of injury, making it difficult to design safe and natural interactions. 
Additionally, the high-frequency noise and mechanical appearance of propeller drones often induce psychological discomfort, further limiting their acceptance in close human proximity\cite{schaffer2021drone-noise-impact,Yeh2017Proxemics} .
To address these issues, previous studies have made proposals such as 
safeguard mechanisms to cover the drone body\cite{Yeh2017Proxemics, Atahi2017touch-based}, 
drones that has an familiar appearance to human\cite{Yeh2017Proxemics}, 
emotion encoding\cite{Cauchard2016emotion-encoding}, 
and bio-inspired propellers that make less noise\cite{noda2018development-of-low-noise-propeller}.

In contrast, flapping-wing drones, inspired by the flight of birds and insects, offer several inherent advantages that make them particularly suitable for physical contact-based interaction\cite{de2020flapping}.  
First, the soft and oscillatory motion of the wings minimizes the physical impact during contact, greatly enhancing physical safety.  
Second, flapping-wing drones produce more natural sound compared to propeller-driven drones, reducing psychological discomfort during interaction.  
Third, the biomimetic appearance and motion of flapping-wing drones evoke a sense of familiarity and warmth, promoting more natural and engaging human-drone interaction.  

Despite these promising characteristics, most existing research on flapping-wing drones has primarily focused on their memchanical characteristics such as aerodynamics, wing design, and flight control\cite{billingsley2021aerodynamic,rifai2008flapping-control,chin2020efficient-flapping}, 
with little attention given to research on the methodlogy for them to interact with humans in a physically and psychologically safe way, especially in scenarios involving direct physical contact.  
To the best of our knowledge, no prior study has specifically investigated the design and implementation of physical contact-based interaction system using flapping-wing drones.  
This presents a significant research gap in leveraging the unique properties of flapping-wing drones to enhance the quality of human-drone interaction.  

Thus, in this study, we propose an interaction system in which a flapping-wing drone performs a palm landing motion on a human hand, enabling direct physical contact in a safe and natural manner.  
To achieve this, we design a trajectory planning method that considers both physical and psychological human factors to facilitate safe and comfortable approaches.  
We implement this method on a flapping-wing drone and conduct real-world experiments to evaluate its palm landing success rate and valiability.  

We focus on palm landing because enabling palm landing has several potential advantages described as follows.
\begin{enumerate}
    \item  It allows environment-adaptive interaction.  
    In crowded or spatially constrained environments, landing on a fixed surface is often impractical.  
    However, by utilizing the human body as a dynamic landing platform, the drone can overcome spatial limitations and operate more flexibly.  
    This approach is particularly useful in urban scenarios, public transportation, or remote field operations.
    \item It contributes to energy efficiency enhancement. 
    Drones have limited battery capacity and typically consume significant energy when hovering.  
    By landing on a human palm during idle periods, the drone can conserve energy and extend its operational time.  
    This energy-efficient operation is critical for long-term service tasks, search and rescue missions, or continuous monitoring operations.
    \item palm landing enhances personalized companion interaction.  
    By physically landing on a human palm, a drone can provide pet-like interaction, evoke emotional attachment, or facilitate social engagement.  
    This is particularly promising for children, the elderly, or individuals with social isolation, where physical interaction fosters a stronger sense of companionship.
\end{enumerate}
These advantages highlight the potential of palm landing as a key interaction modality for flapping-wing drones, enabling a wide range of applications in various scenarios.

The main contributions of this work can be summarized as
follows:
\begin{enumerate}
    \item We propose a human-friendly interaction system utilizing a flapping-wing drone, focusing on trajectory planning.
    \item We develop an intuitive device system for drone interaction.
    \item We verify the proposed model using an actual drone.
\end{enumerate}

The remainder of this paper is organized as follows. 
The basic mecanical characteristics and contorl method of flapping-wing drone is introduced in Section II. 
The motion planning based on physical and psychological factors is presented in Section III,
followed by the experimental results in Section V before concluding in Section VI.
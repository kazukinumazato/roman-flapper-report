Flapping-wing drones have attracted significant attention due to their biomimetic flight.
They are considered more human-friendly due to their characteristics such as low noise and flexible wings, 
making them suitable for human-drone interactions. 
However, few studies have explored the practical interaction between humans and flapping-wing drones. 
On establishing a physical interaction system with flapping-wing drones,
we can acquire inspirations from falconers who guide birds of prey to land on their arms.
This interaction interprets the human body as a dynamic landing platform, which can be utilized in various scenarios such as crowded or spatially constrained environments.
Thus, in this study, we propose a falconry-like interaction system in which a flapping-wing drone performs a palm landing motion on a human hand. 
To achieve a safe approach toward humans, we design a trajectory planning method that considers both physical and psychological factors of the human safety such as the drone's velocity and distance from the user.
We use a commercial flapping platform with our implemented motion planning and conduct experiments to evaluate the palm landing performance and safety.
The results demonstrate that our approach enables safe and smooth hand landing interactions. 
To the best of our knowledge, it is the first time to achieve a contact-based interaction between flapping-wing drones and humans.